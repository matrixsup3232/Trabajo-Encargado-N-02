
\documentclass[preprint,12pt]{elsarticle}

\usepackage[spanish]{babel}
\usepackage{amssymb}
\usepackage{graphicx}
\usepackage{lineno}
\usepackage[utf8]{inputenc}
\usepackage{url}
\usepackage{natbib}
\usepackage{float}
\begin{document}
	


	\begin{frontmatter}

		\title{\huge  DevOps En Base de Datos}
		
		\author{José Edilberto Pastor Mendoza              (2016055237)}
		\author{MAMANI MAMANI, Pedro Luis              (2010038808)}
		\author{PACORA SILVA, Jorge Carlos                   (2013000725)}
		\author{PANTY SIHUAYRO, Juan Carlos               (2014049452)}
		
		\address{Tacna, Perú}
		
		\begin{abstract}
			
Building great software is not just about code. It is also on the management of multiple teams, schedules and frequently Changes, DevOps has become an effective way to help teams collaborate and accelerate the delivery cycle. One of the biggest advantages is development automation and repetitive testing. processes used by database development teams to deliver, manage and maintain the database. From the version that controls changes to deployment in different environments, and when ready, choose to deploy in production, continuous delivery helps teams reduce risk and increase both efficiency and reliability in the software launch process.


		\end{abstract}
\end{frontmatter}

	\section{Resumen}

Construir un gran software no solamente se trata del código. Es también 
sobre la gestión de múltiples equipos, plazos y con frecuencia
Cambios , DevOps se ha convertido en una forma efectiva de ayudar a los equipos a colaborar y acelerar el ciclo de entrega.
Una de las mayores ventajas es la automatización del desarrollo y las pruebas repetitivas.
procesos que utilizan los equipos de desarrollo de bases de datos para entregar, administrar y mantener la base de datos. Desde la versión que controla los cambios hasta la implementación en diferentes entornos, y cuando esté listo, elegir desplegarse en producción, la entrega continua ayuda a los equipos a reducir
arriesgar e incrementar tanto la eficiencia como la confiabilidad en el proceso de lanzamiento del software.


\section{Introduccion}
En DevOps para entornos de prueba de bases de datos, las organizaciones deben generar rápidamente clones del sistema para ofrecer datos que reflejen con precisión el entorno de producción, al tiempo que protegen la información corporativa sensible. Por otro lado, los cambios de la base de datos en la producción representan el abismo más amplio entre las técnicas de desarrollo de aplicaciones ágil y la capacidad de desplegarlas en infraestructuras de TI del mundo real.
CI / CD no es un concepto nuevo y es realmente el punto central de cada implementación de DevOps. He allí que en este articulo nos ocupamos del modo puntual sobre los conceptos basicos de devops, como se desenvuelve en un entorno de base de datos y que herramientas pueden usarse.
	
	
\section{Marco Teorico}
	
\subsection{DEFINICIONES DE DEVOPS}	

Según (Bass, L., Weber, I., Zhu, L., (2015), DevOps A Software Architect’s Perspective, (Edi.) Addison-Wesley Professional, Boston) “DevOps es un conjunto de prácticas destinadas a reducir el tiempo entre el compromiso de un cambio en un sistema y el cambio que se coloca en la producción normal, al tiempo que garantiza una alta calidad.”.

Según (Huttermann, M., (2012), Devops for Developers, (Edi.) Apress, España)” DevOps es una mezcla de patrones destinados a mejorar la colaboración entre desarrollo y operaciones. Direcciones DevOps compartidas Objetivos e incentivos, así como procesos y herramientas compartidos. Porque de los conflictos naturales entre los diferentes grupos, objetivos compartidos y Los incentivos no siempre son alcanzables. Sin embargo, deberían en menos estar alineados unos con otros.”.
\begin{figure}[H]
				\begin{center}
					\includegraphics[width=12cm,height=7cm]{./IMAGENES/devops1}
				\end{center}
Maheta,H.(2018).What is DevOps and Why DevOps:.[Figura].Recuperado de 
https://www.yudiz.com/welcome-devops-prevent-defects/

			\end{figure}



\subsection{¿Qué es la integración continua/distribución continua (CI/CD)?}	
La CI/CD es un método para distribuir aplicaciones de forma frecuente a los clientes mediante el uso de la automatización en las etapas del desarrollo de las aplicaciones. Los principales conceptos que se atribuyen a la CI/CD son la integración continua, la distribución continua y la implementación continua.(Redhat, 2019) 
\begin{itemize}
\item Integración continua 
\item Distribución continua
\end{itemize}
	\begin{figure}[H]
			\begin{center}
					\includegraphics[width=12cm,height=7cm]{./IMAGENES/devops2}
					\includegraphics[width=12cm,height=7cm]{./IMAGENES/devops3}
			\end{center}
			Maheta,H.(2018).What is DevOps and Why DevOps:.[Figura].Recuperado de 
https://www.yudiz.com/welcome-devops-prevent-defects/
		\end{figure}



\subsection{DevOps para bases de datos}	


Adoptar una cultura DevOps le permite implementar códigos más rápido, lo que significa que puede crear, probar y lanzar cambios de software más rápido. Esto, a su vez, lleva a un 21 \% de reducción del trabajo no planificado y de las reelaboraciones y un 19 \% de aumento en los ingresos*. Cuando usted alinee cambios en bases de datos y aplicaciones en un flujo de trabajo integrado, ejecutará funciones de administración de cambios en bases de datos fundamentales dentro de su flujo de trabajo de DevOps sin comprometer la calidad, el rendimiento ni la confiabilidad.


\subsection{La nube para DevOps}

La nube está posicionada de manera única para funcionar como la infraestructura para un abordaje de DevOps por muchas razones, que incluyen:
\begin{itemize}

\item La nube se define por software; es programable y flexible.
\item La nube se acciona por software y tiene muchas capas. El cliente es responsable por cada capa y debe diseñarla, desde la configuración para la cuenta de servicios y credenciales hasta la protección de los datos y los valores de cifrado.
\end{itemize} 


\section{Analisis}
\begin{itemize}
\item Analisis : \\

Praesent sit amet ultrices leo. Nulla semper lacus id metus congue egestas. Proin consectetur tellus eu augue venenatis feugiat. Donec sed ante ut erat sagittis vestibulum. Orci varius natoque penatibus et magnis dis parturient montes, nascetur ridiculus mus. Donec est lacus, efficitur laoreet tempus vitae, mollis tristique tortor. Nulla dolor tellus, fringilla quis facilisis sit amet, dictum maximus purus. In finibus lectus eu tellus pellentesque, varius congue urna molestie. Suspendisse potenti.



\end{itemize}
\section{Conclusion}
\begin{itemize}
\item Conclusion : \\

Las implementaciones de bases de datos pueden ser más rápidas, fáciles y libre de errores al extender las prácticas de DevOps a las bases de datos. Ha demostrado cómo el desarrollo y los equipos de operaciones pueden organizar los procesos de la base de datos para proteger mejor los datos. También ha destacado la importancia de la entrega continua de bases de datos para alentar la organización y los equipos de desarrollo y operaciones para trabajar juntos para construir y entregar un excelente software

\\

Una vez que los equipos tienen la base de datos bajo control de versiones, pueden utilizar más herramientas de Redgate, junto con su servidor de compilación de CI elegido, para integrar y probar continuamente cada cambio confirmado. Esto acelera las versiones para los clientes, reduce el riesgo de problemas de implementación y libera a los desarrolladores de tareas de administración de cambios manuales que consumen mucho tiempo. 

\end{itemize}

	
	\newpage
	
	\bibliographystyle{apalike} 	
	\bibliography{BIBLIOGRAFIA}	 
\citep{referencia01}  
\citep{referencia02}  
\citep{referencia03}  
\citep{referencia04}  
\citep{referencia05}  
\citep{referencia06}  
\citep{referencia07}  
\end{document}

